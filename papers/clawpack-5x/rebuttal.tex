\documentclass{letter}

\usepackage[utf8]{inputenc}
\usepackage{url}

% \usepackage{a4}
% \pagestyle{empty}

\usepackage{color}
\newcommand{\alert}[1]{\textbf{\color{red} #1}}

\begin{document}

\address{Clawpack Developers}
\signature{Kyle T. Mandli, \\
           Aron J. Ahmadia, \\
           Donna Calhoun, \\
           David George, \\
           Yiannis Hadjimichael, \\ 
           David I. Ketcheson, \\
           Grady I. Lemoine, and \\
           Randall J. LeVeque}

\begin{letter}
{
    % PeerJ Editors and Reviewers
}

\opening{To the editors and reviewers of ``The Clawpack 5.X Software'',}

First the authors would like to thank the editor and both reviewers for their
helpful comments.  We feel the suggestions have lead to a more thorough
presentation as well as a cleaner explanation of many points that are crystal
clear to those who are neck deep and completely opaque to others who have not
waded so far into the deep end.

\textbf{\large Editors Comments}
\begin{itemize}
    \item At the moment the title implies the paper is simply a description of
    the software, so it needs changing to something more descriptive. Perhaps
    something along the lines of ``The Clawpack Software: Building an Open
    Source Ecosystem for Solving Nonlinear PDEs''. I would be inclined not to
    put the version number in the tile. \\
    
    We agree with the editor, this title is more reflective of the paper's
    content and we have changed the title accordingly. \\


    \item Page 5, line 4 of sec 2.2: Git submodules are new to me. Can you add a
    brief explanation of what they are? \\

    Additional text and a revision have hopefully provided more detail regarding
    the use of git submodules. \\

    \item Page 6, line 8 of sec 2.3: ``ensuring that past versions of the
    software remain available on a stable and citable platform''. How is that
    done? I'm not sure if you say so later. \\

    Details of this including specifics of the use of Zonodo and DOIs through
    github have been added.
\end{itemize}

The following have been corrected as requested:
\begin{itemize}
    \item Page 1, line -10 of section 1.1: why are both [34], with its URL, and
    the explicit URL given?
    \item Page 1, line -3 of section 1.1: ``with many of the changes from 4.3 to
    4.6'' appears to be part of an incomplete sentence, or at least one
    in need of rewriting.
\end{itemize}


\textbf{\large Reviewer Anders Logg:}
\begin{itemize} 
    \item On p. 2: What is q and what is f? And how does f (typically) depend on
    q (one or two examples). This is clear for most readers but please add one
    or two sentences to give a gentle introduction to non-specialists. \\

    Additional clarification text was added that hopefully illuminates the
    situation. \\

    \item Some of the packages, e.g. PyCLAW on p. 3 and AMRCLAW on p. 4 are
    mentioned before it is made clear on p. 4-5 that these are separate packages
    in the collection that make up CLAWPACK. Please clarify earlier. \\

    The text has been slightly rearranged to avoid this. \\

    \item A diagram could be added to explain the relationships between the
    packages listed on p. 4-5. \\

    We created a draft diagram but decided it was not really all that helpful
    given the space it took up.  Instead we attempted to clarify in the text the
    relationships. \\

    \item The documentation repository is hosted on \url{clawpack.github.com} whereas
    the rest is hosted on \url{github.com/clawpack/}. This looks confusing to
    me. \\

    Clarification on this point was added. \\

    \item Line 198: Should it be ``Travis CI'', not only ``Travis''? Or is
    Travis the team member responsible for running the regression tests? \\

    Instead of forcing a poor graduate student to adopt the name ``Travis'' we
    decided it was wiser to give better clarification on our use of the Travis
    CI service. \\

    \item Figures 2 and 3 are small and very difficult to read. \\

    We have gone back to the original versions and provided \emph{significantly}
    better resolution figures. \\

    \item Remove FIXME/margin note on p. 14. \\

    Yeah, that was stupid, blame the lead author.
\end{itemize}

The following have been corrected as requested:
\begin{itemize}
    \item Add a comma on line 163 (following ``repositories'') for increased
    readability.

    \item Top paragraph on p. 10: Somewhat confusing, seems to contrast AMR with
    MPI (``... on the other hand does not include AMR but uses MPI...''). Please
    reword.
\end{itemize}


\textbf{\large Reviewer Markus Blatt:}

A note regarding the reviewers concern regarding formatting, we were informed
that the editorial staff of PeerJ would assist with formatting issues once the
article was accepted.  If there are outstanding formatting issues beyond what
the PeerJ staff would accommodate please let us know.

\begin{itemize}
    \item The introduction is missing a list of related efforts/software
    packages that are comparable to (parts of) ClawPack. This should be
    added. \\

    A number of similar packages has been added along with appropriate
    citations. \\

    \item Furthermore a section should be added that lists all scientific
    software that is used/needed by ClawPack together with
    references. Currently, only PETSc and matplotlib are cited. According
    to the website at least NumPy is needed/recommend and according to the
    source matlab seems to be also used if it is found. These two should
    be cited. \\

    \alert{ADDRESS THIS} \\

    \item The text in the pictures of Figure 5 cannot be read on a laptop
    screen. Maybe increasing the size of the text or the pictures
    themselves would fix this. Some of the pictures seem to be or use
    third party pictures. The author should check whether their license
    allows the distribution under CC-BY. Some of the pictures (e.g. the
    maps in Figure 3 and 5 might need a copyright attribute added. \\

    Figure 5's resolution has been increased and the text made more clear. 
    Additionally some of the figures have been removed from figures 3 and 5 or
    or distribution rights confirmed and/or cited. \\

    \item lines 27--28 (Introduction): I think it would be nice to list one/some
    of the major changes that will make upgrading to the new user worthwhile or
    make ClawPack more unique. \\

    Text added to address this. \\

    \item lines 563--570 (Conclusion): Similar to above, I am missing a short
    sentence why upgrading is worthwhile for a user. The current conclusion only
    takes into account the view of developers. \\

    Text in the conclusion added to also reiterate this. \\

    \item lines 216 - 223:  In the abstract it says that clawpack has been
    developed as open source for 20 years. Here it reads like the software has
    been made open source recently as you expect it to increase the developer
    community. Please clarify! \\

    Hopefully this has been clarified via the additional text. \\

    \item line 358:  ``..., where some operation ...'' Is it really some (might be
    different per thread) or the same operation? \\

    \alert{ADDRESS THIS!} \\

    \item lines 357 - 362:  ``The main ... patch at a time'' Application of
    \texttt{parallel\_for} is not clear to me. First it seems like
    parallelization will be over the patches. Then you say that each iteration
    corresponds to a grid level (shouldn't that be a patch?). Afterwards you say
    you assign one patch to each thread. It might also be that my understanding
    of a grid is different from the author's. Please clarify, currently the
    parallelization strategy is not clear to me. \\

    Additional text has been added that hopefully clarifies this point. \\

    \item lines 370 - 371: 40 cores or 20 cores/40 threads? Which exact processor
    model did you use? \\

    Text has been added and clarification about the testing hardware was added.
    \\

    \item line 376: ``The efficiency is above 80\% until 24 cores, then drops...'' 
    I would have expected it drop for >20 threads as two threads share one core.
    Might the preservation of efficiency be due to the ordering of the patch by
    size? Do you have an explanation? \\

    The test hardware does not use hyperthreading and does contain 40 cores. 
    Text to clarify this was added. \\

    \item lines 377--380:  I don't understand these sentences about level n grids.
    Does two level 1 grids mean that 2 cells on level 0 got refined? Maybe the
    meaning/definition of a level n grid could/should be stated. \\

    Clarification text was added. \\

    \item Figure 3:  The text in the left two pictures is hardly readable on a
    laptop screen. Maybe making the picture larger would help. If these are 3rd
    party pictures, please check whether they can be distributed via CC-BY
    License. At least for the map a reference is missing. \\

    Figure 3 has been modified to remove questionable copyrighted material and
    made higher resolution. \alert{CHECK THIS!} \\

    \item Figure 5:  Please double check whether you are allowed to distribute
    the pictures under CC-BY License and whether they need a reference to be
    added. \\

    Distribution rights were checked and addressed. \\

    \item lines 480-482: Is the definition of patch here the same as in the
    description of the OpenMP parallelization above (lines 357--362)? If this is
    the case then the mesh-patch equivalence should be mentioned there. \\

    Clarification text regarding the patches vs. mesh was added. \\

    \item Page 14:  There is an annotation in the original PDF. \\

    Yeah, again, blame the lead author on that one.

\end{itemize}

The following have been corrected as requested:
\begin{itemize}
    \item The referencing style for figures should be used consistently. Figure
    1 is referenced as ``Figure 1'' while the rest is referenced as ``Fig. 2'',
    etc.

    \item lines 49--50: Reference [34] contains the URL. Therefore the last
    sentence (lines 49-50) is superfluous.

    \item line 227: Citation of PETSc reference missing.

    \item line 354: ``GEOCLAW. where'' has a superfluous dot.

    \item line 368: ``Figure 1'' -> ``Fig. 1'' for consistency.

    \item line 377: ``Note that there are only 2 level 1'' -> ``Note that there
    are only two level 1''

    \item line 382: Citation of PetSc missing.
\end{itemize}

\begin{itemize}
    \item Mention improvement of AMR discussion
    \begin{itemize}
        \item Fixed some of the figures
        \item Added more explanation on AMR and added figure
        \item Parallel explanations
    \end{itemize}
    \item Added some open source details
    \item Cleaned up parallelism architectures
    \item Cite PETSc - line 227, all PETSc references are included
    \item Add text regarding ecosystem
    \item Diagram of package dependency - added better explanation in text
    rather than diagram (tried, did not seem to work well)
\end{itemize}


\vspace{4cm}

\closing{Sincerely,}

\end{letter}
\end{document}

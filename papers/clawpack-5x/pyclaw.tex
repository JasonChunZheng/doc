%!TEX root = paper.tex
%
% PyClaw
%
% Lead currently:  David Ketcheson
%

\subsection{\pyclaw}

Later 4.x releases included a number of Python-based tools for handling
Clawpack input and output.  The 5.0 release includes a full-fledged Python
solver in which the higher-level parts of Clawpack have been reimplemented in
Python.  This new solver also includes access to the high-order algorithms
introduced in SharpClaw and can be used on large distributed-memory parallel
machines.  Lower-level code (whatever gets executed repeatedly and needs to be
fast) from the earlier Fortran Classic and SharpClaw codes is automatically
wrapped at install time using f2py.

\subsubsection{Librarization and extensibility}
Scientific software is easier to use, extend, and integrate with other tools when
it is designed as a library \cite{brown2014run}.  Clawpack has always been designed
to be extensible, but PyClaw takes this further in several ways.  First, it is
distributed via a widely-used package management system, pip.
Second, the default installation process ("pip install clawpack")
provides the user with a fully-compiled code and does not require setting environment
variables.  Like other Clawpack packages, PyClaw provides several "hooks" for users
to plug in custom routines (for instance, to specify boundary conditions).
In PyClaw, these routines -- including the Riemann solver itself -- are selected at
run-time, rather than at compile-time.  These routines can be written directly in
Python, or (if they are performance-critical) in a compiled language (like Fortran or C)
and wrapped with one of the many available tools.  Problem setup (including things like
initial conditions, algorithm selection, and output specification) is also
performed at run-time, which means that researchers can bypass much of the slower
code-compile-execute-postprocess cycle.
It is intended that PyClaw be easily usable within other packages (without control of main()).

\subsubsection{Python classes for data on collections of structured grids}
PyClaw includes Python classes for describing collections of structured grids
and data on them.
These classes are also used by the other codes through VisClaw, for post-processing.

Briefly describe: Dimension, Grid, Patch, State, Solution, Controller.

\subsubsection{PyClaw Solvers}
\begin{itemize}
    \item Classic
    \item SharpClaw: new things include WENO (up to 17th order and in 3D), RK and multistep
            time integration with rigorous SSP timestepping (Yiannis)
\end{itemize}

\subsubsection{PyClaw backends}
\begin{itemize}
    \item pure numpy
    \item PETSc
    \item Boxlib (Matt)
\end{itemize}

\begin{itemize}
    \item overall structure/languages figure
    \item IPython notebooks
\end{itemize}

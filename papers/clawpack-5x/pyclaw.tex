%!TEX root = paper.tex
%
% PyClaw
%
% Lead currently:  David Ketcheson
%

\subsection{\pyclaw} \label{sec:pyclaw}

Later 4.x releases included a number of \todo{I think some version of a full
solver existed in 4.x but was primitive compared to the 5.x version}{Python-
based tools} for handling \clawpack input and output.  The 5.x releases include
a set of full-fledged Python solvers in which the higher-level parts of
\clawpack have been reimplemented in Python and adapted to an object-oriented
framework.  This new solver also includes access to the high-order algorithms
introduced in SharpClaw (i.e., WENO reconstruction and Runge--Kutta integrators)
and can be used on large distributed-memory parallel machines.  Lower-level code
(whatever gets executed repeatedly and needs to be fast) from the earlier
Fortran Classic and SharpClaw codes is automatically wrapped at install time
using \texttt{f2py}.

\pyclaw's data structures and IO routines are also used by the other \clawpack
codes and VisClaw for post-processing.

\subsubsection{Librarization and Extensibility}
Scientific software is easier to use, extend, and integrate with other tools when
it is designed as a library \cite{Brown:2015cj}.  \clawpack has always been designed
to be extensible, but \pyclaw takes this further in several ways.  First, it is
distributed via a widely-used package management system, \texttt{pip}.
Second, the default installation process (``\texttt{pip install clawpack}'')
provides the user with a fully-compiled code and does not require setting environment
variables.  Like other \clawpack packages, \pyclaw provides several ``hooks'' for users
to plug in custom routines (for instance, to specify boundary conditions).
In \pyclaw, these routines -- including the Riemann solver itself -- are selected at
run-time, rather than at compile-time.  These routines can be written directly in
Python, or (if they are performance-critical) in a compiled language (like Fortran or C)
and wrapped with one of the many available tools.  Problem setup (including things like
initial conditions, algorithm selection, and output specification) is also
performed at run-time, which means that researchers can bypass much of the slower
code-compile-execute-post-process cycle.
It is intended that \pyclaw be easily usable within other packages (without control of \texttt{main()}).  This is in contrast to \amrclaw, \geoclaw and classic, which provides an end-to-end solution.

% \subsubsection{Python classes for data on collections of structured grids}
% \pyclaw includes Python classes for describing collections of structured grids
% and data on them.
% These classes are also used by the other codes through VisClaw, for post-processing.

% A mesh in \clawpack always consists of a set of (possibly) mapped tensor-product
% grids (interval, quadrilateral, or hexahedral).  In \pyclaw this geometry is represented
% through the following objects:
% \begin{itemize}
%     \item A {\bf Dimension} is an interval divided into a fixed number of equally-sized subintervals.
%     \item A {\bf Patch} is the tensor product of 1-3 Dimension objects, possibly with an associated mapping.
%     \item A {\bf Domain} is a collection of any number of possibly overlapping Patch objects.
% \end{itemize}
% At present, \pyclaw solvers operate only on a single patch.  Multi-patch capability
% is used to represent geometries from \amrclaw and \geoclaw runs, and is planned for
% utilization in a future AMR-enabled version of \pyclaw.

% The simulation data are represented by an additional set of objects:
% \begin{itemize}
%     \item A {\bf State} contains values of the solution vector $q$ and any
%         auxiliary variables for each cell in a given Patch.
%     \item A {\bf Solution} is a collection of States, corresponding to values of the solution
%         on each Patch in a Domain.
% \end{itemize}

% \subsubsection{\pyclaw Solvers}
% A Solver is used to advance a Solution in time.  Two types of solvers are implemented:
% \begin{itemize}
%     \item {\bf Classic}: the Lax-Wendroff-based solvers that are used in other Clawpack codes.
%     \item {\bf SharpClaw}: high-order solvers based on WENO reconstruction and Runge-Kutta or linear multistep time integration.
% \end{itemize}
% All \pyclaw Solvers can use both fixed and variable step-size.
% In order to preserve stability a CFL-like condition is being checked at each step of the computation, based on
% a user-defined maximum CFL number.
% The step-size must be chosen in such a way that the CFL number is always less than the maximum allowed 
% value, thus in the case of variable step-size a rigorous step-size control is required.
% We distinguish the Lax-Wendroff-based solvers and Runge-Kutta methods from linear multistep methods,
% since the step-size strategy is different.

% For the Lax-Wendroff and Runge-Kutta methods, first a choice for the step-size is made according to the 
% maximum wave speed of the current solution 
% (and the maximum wave speeds of stage solutions for Runge--Kutta methods).
% Then a new solution is computed using this step-size and the CFL number is evaluated.
% If the CFL number is less than the maximum allowed value then the step is accepted and a new step-size is 
% chosen in the same manner.
% Otherwise, the wave speed at the current step is large enough to cause instability or spurious oscillations if 
% such a step-size is used, hence the step is rejected.
% An updated step-size based on the increase of the maximum wave speed is calculated and the step is
% repeated.
% This guarantees that a step is rejected at most once and in most cases only the first step is rejected due to 
% large initial step-size.
% Currently SharpClaw Runge--Kutta time integrators consist of Euler's method, second-, third- and forth-order
% optimal SSP Runge--Kutta methods with two, three and ten stages respectively.
% Also a general Runge-Kutta method with user-defined coefficients can be used. 

% For linear multistep methods the step-size selection is more involved since a variable step-size depends on
% the SSP coefficient of the method which in turn depends on previous step-sizes and the method's coefficients 
% that are varied to attain hight order. 
% All these intricate relations are analysed in \todo{cite SSPVLMM paper}, where second and third order SSP linear
% multistep methods with variable step-size are developed.
% SharpClaw linear multistep methods consist of second-order, three-step and third-order, four-step optimal 
% SSP linear multistep methods.
% In both cases explicit formula for the step-size are used based on the permissible SSP step-size that 
% guarantees the SSP property.
% SSP Runge--Kutta are used for the starting values and the step-size for the those steps is chosen as
% explained above.
% For the third-order linear multistep methods additional {\em a posteriori} conditions must be applied.
% Basically, they imply that the step-size remains finite and non-zero and that the forward Euler step-size
% changes little over a single step.
% Whenever these conditions are violated the step is rejected, a new step-size is derived and the solution is
% recalculated.
% Again, the fix in the step-size is such that the step is rejected at most once.

% SharpClaw uses a traditional method-of-lines approach, in which
% spatial operators are discretized first, to achieve high-order
% resolution in space and time.  Spatial derivaties are discretized
% using a high-order nonoscillatory reconstruction of the solution from
% cell averages, which results in a system of ODEs that are subsequently
% solved using the time-stepping methods described above.  The spatial
% reconstructions are obtained using the weighted essentially
% nonoscillatory (WENO) reconstruction algorithm which suppresses
% spurious oscillations near discontinuities.  The WENO routines in
% SharpClaw were generated by PyWENO, which is a standalone package for
% building custom WENO codes. \cite{ketcheson2012pyclaw} \todo{should a
%   breif description of PyWENO appear here, or just a citation?}


\subsubsection{PyClaw backends}
\begin{itemize}
    \item pure numpy
    \item PETSc
    \item Boxlib (Matt)
\end{itemize}

\begin{itemize}
    \item overall structure/languages figure
    \item IPython notebooks
\end{itemize}

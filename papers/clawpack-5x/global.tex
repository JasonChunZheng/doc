%!TEX root = paper.tex
%
% Global Changes
%
% Lead currently:  None
%

\subsection{Global Changes}
Substantial redesign of the \clawpack code base was performed in the move
from \clawpack 4.x to 5.x.  A summary of these changes (and changes within
the 5.x line) can be found at \url{http://www.clawpack.org/clawpack5.html}.
Major changes that affected all aspects of the code include:
\begin{itemize}
    \item The interface to the Riemann solvers was changed so that the same Fortran
    Riemann solvers can be used for all versions of the code (including \pyclaw
    via \texttt{f2py}).  Rather than appearing in scattered example directories,
    these Riemann solvers have all been collected into the new
    \texttt{riemann} repository and calling sequences were modified.  Calling
    sequences for a number of other Fortran subroutines were also modified based
    on experiences with the \clawpack 4.x code.  These can also be used as a
    stand-alone product for those who only want the Riemann solvers.
    \item Python front-ends were redesigned for specifying run-time options for
    the solver (\texttt{setrun.py} and visualization \texttt{setplot.py}.  The
    Fortran variants (\classic, \amrclaw, and \geoclaw) all use a Python script
    to facilitate setting input variables.  This creates text files with a
    rigidly specified format that are then read in when the Fortran code is run.
    The interface now allows updates to the input parameters while maintaining
    backwards compatibility.
    \item The indices of the primary conserved quantities were reordered.  In
    \clawpack 4.x, the $m$th component of a
    system of equations in grid cell $(i,j)$ (in two dimensions, for example),
    was stored in \texttt{q(i,j,m)}.  In order to improve cache usage and to
    more easily interface with PETSc, a global
    change was made to the ordering so that the component number comes
    first; i.e. \texttt{q(m,i,j)}.  A seemingly minor change like this affects a
    huge number of lines in the code and cannot easily be automated. The use of
    version control and regression tests was crucial in the successful
    completion of the project.
\end{itemize}

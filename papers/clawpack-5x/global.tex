%!TEX root = paper.tex
%
% Global Changes
%
% Lead currently:  None
%

\subsection{Global Changes}
Substantial redesign of the Clawpack code base was performed in the move
from Clawpack 4.x to 5.x.  A summary of these changes (and changes within
the 5.x line) can be found at \url{http://www.clawpack.org/clawpack5.html}.
Major changes that affected all aspects of the code include:
\begin{itemize}
    \item Reordering indices.  In Clawpack 4.x, the $m$th component of a
system of equations in grid cell $(i,j)$ (in two dimensions, for example),
was stored in \texttt{q(i,j,m)}.  This index order caused problems in
optimizing code and when interfacing with PETSc in PyClaw, and so a global
change was made to change the ordering so that the component number comes
first, e.g. \texttt{q(m,i,j)}.  A seemingly minor change like this affects a
huge number of lines in the code and cannot easily be automated. The use of
version control and regression tests was crucial in the successful
completion of the project (although bugs are still surfacing in infrequently
exercised parts of the code).
    \item Calling sequences for the Riemann solvers.  The same Fortran
Riemann solvers can be used for all versions of the code (including \pyclaw
thanks to \texttt{f2py}).  These have all be collected into the new
\texttt{riemann} repository and calling sequences were modified.  Calling
sequences for a number of other Fortran subroutines were also modified based
on experiences with the Clawpack 4.x code.
    \item Input variables in \texttt{setrun.py}.  The Fortran variants
(\classic, \amrclaw, and \geoclaw) all use a Python script to facilitate
setting input variables, typically named \texttt{setrun.py}.  This is
pre-processed to create text files with a rigidly specified format that are
then read in when the Fortran code is run.  Input parameter options had
evolved over the years to the point where a redesign of this interface was
required.
\end{itemize}

%!TEX root = paper.tex
%
% Global Changes
%
% Lead currently:  None
%

\subsection{Global Changes}
Substantial redesign of the \clawpack code base was performed in the move
from \clawpack 4.x to 5.x.
Major changes that affected all aspects of the code include:
\begin{itemize}
    \item The interface to the \clawpack Riemann solvers was changed so that one
    set of solvers can be used for all versions of the code (including \pyclaw
    via
    \texttt{f2py}\footnote{\url{http:\\docs.scipy.org/doc/numpy-dev/f2py}}).
    Rather than appearing in scattered example directories,
    these Riemann solvers have all been collected into the new
    \texttt{riemann} repository. Modifications to the
    calling sequences were made to accommodate this increased generality.
    \item Calling sequences for a number of other Fortran subroutines were also
    modified based on experiences with the \clawpack 4.x code.  These can also
    be used as a stand-alone product for those who only want the Riemann
    solvers.
    \item Python front-ends were redesigned to more easily
    specify run-time options for
    the solver and visualization.  The Fortran variants (\classic, \amrclaw, and
    \geoclaw) all use a Python script to facilitate setting input variables.
    These scripts create text files with a rigidly specified format that are then read
    in when the Fortran code is run. The interface now allows updates to the
    input parameters while maintaining backwards compatibility.
    \item The indices of the primary conserved quantities were reordered.  In
    \clawpack 4.x, the $m$th component of a
    system of equations in grid cell $(i,j)$ (in two dimensions, for example),
    was stored in \texttt{q(i,j,m)}.  In order to improve cache usage and to
    more easily interface with PETSc, a global
    change was made to the ordering so that the component number comes
    first; i.e. \texttt{q(m,i,j)}.  A seemingly minor change like this affects a
    huge number of lines in the code and cannot easily be automated. The use of
    version control and regression tests was crucial in the successful
    completion of the project. 
    %\comment{MJB Interestingly, we did not observe any appreciate change in performance.}
\end{itemize}

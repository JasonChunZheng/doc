%!TEX root = paper.tex
%
% Development Approach
%
% Lead currently:  Aron Ahmadia's 
%

\section{Development Approach}

\subsection{Github organization, subrepositories}
Clawpack includes several related software projects, each of which is managed
in its own repository on Github.  These projects are sub-repositories of the
high-level Clawpack repository and are assigned to the Clawpack Github
organization.  The set of maintainers -- with push and merge privileges -- is
different for each sub-repository.

The main solver repositories are:
\begin{itemize}
    \item Riemann (Riemann solvers used by all the other projects)
    \item Classic
    \item \amrclaw
    \item \geoclaw
    \item \pyclaw
\end{itemize}

Additional repositories contain documentation and extended examples of using the code:
\begin{itemize}
    \item doc
    \item apps
\end{itemize}

\subsection{Development Model}

\subsubsection{Contributing}
Scientist programmers are often discouraged from sharing code
due to existing reward mechanisms and the fear of being "scooped".
In fact, scientific communities that openly share and develop code
have an advantage because each researcher can leverage the work of
many others \cite{turk2013scaling}.

Over the past twenty years, a great number of users have written
additional code, extending Clawpack with new Riemann solvers,
algorithms, and domain-specific problem tools.  Most of this code
has not made it back into the core library.  With Clawpack 5.x,
we are trying to encourage contributions from a broad community, with
tools like distributed version control and open discussions on 
the mailing lists and issue trackers.
Clawpack is supported by a community of user-developers whose
collaboration

Clawpack uses the distributed version control software git.
The main repository is hosted on Github.  Anyone may contribute,
for instance by developing new code, reporting bugs, or suggesting
improvements.

Bugs and feature requests are posted as issues on the tracker that
is part of the Github repository.  The tracker provides a page for
discussing the issue.

The primary development model
is typical for Github projects: a contributor forks the repository on Github,
and develops improvements in a branch that is pushed to her own fork.
She issues a "pull request" (PR) when the branch is ready to be merged
into the main repository.  Increasingly, contributors are also using
PRs as a way to conveniently post preliminary or prototype code for
discussion prior to further development.

After a PR is issued, other developers -- including one or more of the
maintainers for the corresponding repository -- reviews the code.  The Travis
CI server also automatically runs the tests on the proposed new code.  The test
results are visible on the Github page for the PR.  Usually there is some
iteration as developers suggest improvements, request more testing, etc.
Once the tests are passing and it is agreed that the code is acceptable, a
maintainer merges it.

\subsubsection{Releases}

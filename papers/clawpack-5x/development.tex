%!TEX root = paper.tex
%
% Development Approach
%
% Lead currently:  Aron Ahmadia's
%

\section{Development Approach} \label{sec:development}

\clawpack's development model is driven by the needs of its
developer community.  The \clawpack project consists of several
interdependent projects: core solver functionality, a
visualization suite, a general adaptive mesh refinement code, a
specialized geophysical flow code, and a massively parallel Python
framework.  Changes to the core solvers and visualization suite have a
downstream effect on the other codes, and the developers largely work
in an independent, asynchronous manner across continents and time
zones.

\vskip 5pt
The core \clawpack software repositories are:
\begin{itemize}
    \item \texttt{clawpack} -- responsible for installation and coordination of
    other repositories,
    \item \texttt{riemann} -- Riemann solvers used by all the other
    projects,
    \item \texttt{visclaw} -- a visualization suite used by all the other
    projects,
    \item \texttt{clawutil} -- utility functions used by most other
    projects,
    \item \texttt{classic} -- the original single grid methods in 1, 2, and 3
    space dimensions,
    \item \texttt{amrclaw} -- the general adaptive mesh refinement
    framework in 2 and 3 dimensions,
    \item \texttt{geoclaw} -- solvers for depth-averaged
    geophysical flows which employs the framework in \texttt{amrclaw}, and
    \item \texttt{pyclaw} -- a Python implementation and interface to the
    \clawpack algorithms including high-order metods and massively
    parallel capabilities.
\end{itemize}

\noindent
A release of \clawpack downloaded by users contains all of the above.
The repositories \texttt{riemann}, \texttt{visclaw}, and
\texttt{clawutil} are sometimes referred to as \textit{upstream}
projects, since their changes affect all the remaining projects in the
above list, commonly referred to as \textit{downstream} projects.
There are some variations on this, for instance \amrclaw is upstream
of \geoclaw, which uses many of the algorithms and software base from
\amrclaw.  To coordinate this the \texttt{clawpack} repository
%records broad changes that have multi-repository implications.
points to the compatible version of each repository, described later
in this section.

Beyond the major code containing repositories additional repositories contain
documentation and extended examples for using the packages:
\begin{itemize}
    \item \texttt{doc} -- the primary documentation source files,
    developed using Sphinx\footnote{\url{http://sphinx-doc.org}},
    \item \texttt{clawpack.github.com} -- a host repository for the
    documentation html files
    that appear at \url{http://www.clawpack.org}, and
    \item \texttt{apps} -- applications contributed by developers and
    users that go beyond the introductory examples included in the core
    repositories.
\end{itemize}
The Clawpack 4.x code is also available in the repository
\begin{itemize}
    \item \texttt{clawpack-4.x}
\end{itemize}
but is no longer under development.


\subsection{Version Control}

The \clawpack team uses the Git distributed version control system
to coordinate development of each major project.  The repositories are
publicly coordinated under the \clawpack organization on
GitHub\footnote{\url{https://github.com/clawpack}} with the
top-level \texttt{clawpack} super-repository responsible for hosting
build and installation tools, as well as providing a synchronization
point for the other repositories.  The remaining ``core \clawpack repositories''
listed above are subrepositories of the main \texttt{clawpack} organization.

GitHub itself is a free provider of public Git repositories.  In addition to
repository hosting, the \clawpack team uses GitHub for issue tracking,
code review, automated continuous integration via Travis CI\footnote{\url{https://travis-ci.org/}},
and test coverage tracking via Coveralls\footnote{\url{http://coveralls.io}}
for the Python-based modules.  The issue tracker on
GitHub supports cross-repository references,
simplifying communication between \clawpack developer sub-teams.  The
Travis CI service, which provides free continuous integration for
publicly developed repositories on GitHub, runs \clawpack's test
suites through \texttt{nose}\footnote{\url{https://nose.readthedocs.org}}
on proposed changes
to the code base, and through a connection to the Coveralls service,
reports on any test failures as well as changes to test coverage.

\subsection{Submodules}

The \texttt{clawpack} ``super-repository'' serves as an
installation and synchronization point for the project repositories:
each of the other core \clawpack repositories listed above is a submodule
of the \texttt{clawpack} repository.  A commit to the \texttt{clawpack}
repository serves to keep track of the
versions of each submodule that are meant to function together.

Typically the \clawpack developers advance the master development
branch of the top-level \texttt{clawpack} 
repository any time a major feature is added
or a bug is fixed in one of the upstream projects that might affect code in
other repositories.  By checking out a particular commit in the
\texttt{clawpack} repository and performing a \texttt{git submodule update},
all repositories can be updated to versions that are intended to be
consistent and functional.

In particular, when Travis runs the regression tests in any project
repository (performed automatically for any pull request), it starts
by installing \clawpack on a virtual machine and the current head
of the \texttt{clawpack/master} branch indicates the commit from each of the
other projects that must be checked out before performing the tests.
If the \texttt{clawpack} repository has not been properly updated
following changes in other upstream projects, these tests may fail.

Any new release of \clawpack is a snapshot of one particular commit on
\texttt{clawpack} and the related commits on all submodules.  These
particular commits are also tagged for future reference with consistent
names, such as \texttt{v5.3.1}.  (Git tags simply provide a descriptive name
for a particular commit rather than having to refer to a Git hash code.)

\ignore{
Snapshots of the top-level repository are taken both to prepare for
software releases of the \clawpack project itself, and is used to provide
``known working states'' when changes from upstream repositories must
be coordinated with their downstream consumers.
%\comment{MJB - should describe a little more how}

When a feature is
added to an upstream repository that affects other repositories, the
addition of this change is coordinated with the downstream
repositories by simultaneously advancing the version of the upstream
repository as well as all affected downstream repositories.
}

Git submodules provide an invaluable
mechanism for allowing \clawpack team members to work asynchronously on
independent projects while reusing and maintaining common software
infrastructure.


\subsection{Contributing}

Scientists who program are often discouraged from sharing code
due to existing reward mechanisms and the fear of being ``scooped''.
However, recent studies indicate that 
scientific communities that openly share and develop code
have an advantage because each researcher can leverage the work of
many others \cite{Turk:2013hd}, and that paper citation rates can be
increased by sharing code \cite{Vandewalle2012} and/or 
data \cite{PiwowarDayEtAl2007}.  
Moreover, journals and funding agencies are increasingly requiring
investigators to share code used to obtain published results.  One
of the goals of the \clawpack project is to facilitate code sharing
by users, by providing an easy mechanism to refer to a specific
version of the \clawpack software and ensuring that past versions
of the software remain available on a stable and citable platform.

On the development side, we expect that the open source development model
with important discussions conducted in public will lead to further growth of
the developer community and additional contributions from users.
Over the past twenty years, many users have written
code extending \clawpack with new Riemann solvers, algorithms, or
domain-specific problem tools.  Unfortunately, much of this code did not
make it back into the core software for others to use.
Many of the development changes in \clawpack 5.x were done
to encourage contributions from a broader community. We have begun to see an
increase in contributions from outside the developers' groups, and hope to
encourage more of this in the future.

%through tools such as
%distributed version control and open discussions on the mailing lists and issue
%trackers.

The primary development model is typical for GitHub projects: a
contributor forks the repository on GitHub, then develops improvements
in a branch that is pushed to her own fork.  She issues a ``pull
request'' (PR) when the branch is ready to be merged into the main
repository.  Increasingly, contributors are also using PRs as a way to
conveniently post preliminary or prototype code for discussion prior
to further development, often marked WIP for ``work in progress'' to signal
that it is not ready to merge.

After a PR is issued, other developers, including one or more of the
maintainers for the corresponding project, review the code.  The Travis
CI server also automatically runs the tests on the proposed new code.  The test
results are visible on the GitHub page for the PR.  Usually there is
some iteration as developers suggest improvements or discuss
implementation choices in the code.  Once the tests are passing and it
is agreed that the code is acceptable, a maintainer merges it.

\subsection{Releases}

Although \clawpack is continuously developed, it is convenient for
users to be able to install stable versions of the software.  The
\clawpack developers provide these releases through two distribution
channels: GitHub and the Python Package Index (PyPI).  Full source
releases are available on GitHub.  Alternatively, the PyClaw
subproject and its dependencies can be installed automatically using a
PyPI client such as \texttt{pip}.

\clawpack does not follow a calendar release cycle.  Instead, releases
emerge when the developer community feels enough changes have
accumulated since the last release to justify
the cost of switching to a new release.  For the most part, \clawpack
releases are
versioned using an $M.m.p$ triplet, representing the major (M), minor (m), and
patch (p) versions respectively.  In the broader software engineering
community, this is often referred to as semantic versioning.  Small
changes that fix bugs and cosmetic issues result in increments to the
patch-level.  Backwards-compatible changes result in an increase to
the minor version.  The introduction of backwards-incompatible changes
require that the major version be incremented.  In addition, the
implementation of significant new algorithms or capability will also
justify the increment of major release number, and is often an impetus
for providing another release to the public.
In practice, the \clawpack software has frequently included changes in minor
version releases that were not entirely backwards compatible, but these have
been relatively minor and documented in the release notes.  Major version
numbers have changed infrequently and related to major refactoring of the
code as in going from 4.x to 5.0.

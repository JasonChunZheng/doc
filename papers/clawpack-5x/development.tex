%!TEX root = paper.tex
%
% Development Approach
%
% Lead currently:  Aron Ahmadia's 
%

\section{Development Approach}

\clawpack's development model is largely driven by the needs of its
developer community.  The \clawpack project contains core solver
functionality, a visualization suite, a general adaptive mesh
refinement code, a specialized geophysical flow code, and a
massively parallelized Python framework.  Changes to the core solvers
and visualization suite have a downstream effect on the other codes,
and the developers largely work in an independent, asynchronous manner
across continents and time zones. 

\subsection{GitHub}

The \clawpack team leverages the Git distributed version control system
to coordinate development, with each major project and its development
team assigned to its own repository.  The repositories are each
publicly coordinated under the \clawpack organization on GitHub, 
\url{https://github.com/orgs/clawpack/dashboard} and a
top-level \texttt{clawpack} repository is responsible for hosting
build and installation tools as well as providing a synchronization
point for the other repositories.

GitHub is a free provider of public Git repositories.  In addition to
repository hosting, the Clawpack team uses GitHub for issue tracking,
code review, automated continuous integration via Travis CI \cite{travis}, 
and test coverage tracking via Coveralls \cite{coveralls}.  
The issue tracker on GitHub
automatically recognizes cross-repository references, simplifying
communication between \clawpack developer sub-teams.  The Travis CI
services, which provides free continuous integration for publicly
developed repositories on GitHub, runs \clawpack's test suites on
proposed changes to the code base, and through a connection to
the Coveralls service, reports on any test failures as well as changes
to test coverage.

\subsection{Submodules}

As mentioned earlier, the \texttt{clawpack} repository serves as an
installation and synchronization point for the other repositories.
From a Computer Science perspective, the \texttt{clawpack} repository
holds \textit{references} to a version of the code in each
repository.  

The main \clawpack repositories are:
\begin{itemize}
    \item \texttt{clawpack} - Installation, coordination of other repositories,
    \item \texttt{riemann} - Riemann solvers used by all the other projects,
    \item \texttt{visclaw} - Visualization suite used by all the other projects,
    \item \texttt{clawutil} - Utility functions used by most other projects,
    \item \texttt{classic} - Uniform grid methods in 1, 2, and 3 space dimensions,
    \item \texttt{amrclaw} - Adaptive mesh refinement (AMR) in 2 and 3 dimsensions,
    \item \texttt{geoclaw} - AMR codes for shallow geophysical flows,
    \item \texttt{pyclaw} - Python implementation and interface to the \clawpack algorithms including massive parallelism capabilities
\end{itemize}

Additional repositories contain documentation and extended examples of
using the code: 
\begin{itemize}
    \item \texttt{doc} - Documentation
    \item \texttt{apps} - Applications contributed by developers and users
\end{itemize}
Clawpack 4.x is still available in the repository
\begin{itemize}
    \item \texttt{clawpack-4.x}
\end{itemize}
but is no longer under active development.

We sometimes refer to \texttt{riemann}, \texttt{visclaw}, and
\texttt{clawutil} as
\textit{upstream} repositories, since their changes affect the
remaining repositories in the list, which we usually refer to as
\textit{downstream} repositories.

Typically, the \clawpack developers advance the master development
branch of the repository any time a major feature is added or a bug is
fixed in one of the other repositories.  When a feature is added to an
upstream repository that affects other repositories, the addition of
this change is coordinated with the downstream repositories by
simultaneously advancing the version of the upstream repository as
well as all affected downstream repositories.  

Git submodules are very powerful, but many developers, both new and
experienced, find them confusing and challenging to work with.
However, they have proved invaluable in allowing the \clawpack team to
work asynchronously on sub-projects while reusing and maintaining
common software infrastructure.

\subsection{Contributing}

\todo{Revise Contributing section, remove redundancy}

Scientist programmers are often discouraged from sharing code
due to existing reward mechanisms and the fear of being ``scooped''.
In fact, scientific communities that openly share and develop code
have an advantage because each researcher can leverage the work of
many others \cite{Turk:2013hd}.

Over the past twenty years, a great number of users have written
additional code, extending \clawpack with new Riemann solvers,
algorithms, and domain-specific problem tools.  Most of this code
has not made it back into the core library.  With \clawpack 5.x,
we are trying to encourage contributions from a broad community, with
tools like distributed version control and open discussions on 
the mailing lists and issue trackers.
\clawpack is supported by a community of user-developers whose
collaboration is enabled by the use of git and Github.

% \clawpack uses the distributed version control software Git.
% The main repository is hosted on GitHub.  Anyone may contribute,
% for instance by developing new code, reporting bugs, or suggesting
% improvements.

% Bugs and feature requests are posted as issues on the tracker that
% is part of the Github repository.  The tracker provides a page for
% discussing the issue.

The primary development model
is typical for Github projects: a contributor forks the repository on Github,
and develops improvements in a branch that is pushed to her own fork.
She issues a ``pull request'' (PR) when the branch is ready to be merged
into the main repository.  Increasingly, contributors are also using
PRs as a way to conveniently post preliminary or prototype code for
discussion prior to further development.

After a PR is issued, other developers -- including one or more of the
maintainers for the corresponding repository -- reviews the code.  The Travis
CI server also automatically runs the tests on the proposed new code.  The test
results are visible on the Github page for the PR.  Usually there is some
iteration as developers suggest improvements, request more testing, etc.
Once the tests are passing and it is agreed that the code is acceptable, a
maintainer merges it.

\subsection{Releases}

\todo{Add release information}


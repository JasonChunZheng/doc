%!TEX root = paper.tex
%
% GeoClaw
%
% Lead currently:  Randy LeVeque
%

\subsection{\geoclaw} 

The \geoclaw branch of \clawpack originated with the work of George
\cite{dgeorge:masters, dgeorge:phd, dgeorge:jcp} 
%\cite{george:2006, George:2008aa} 
on solving the
two-dimensional shallow water equations using adaptive mesh refinement under
conditions necessary for modeling tsunami generation, propagation, and
inundation.
The \amrclaw code formed the starting point but it was necessary to make many 
modifications to support the requirements of this application, as described
briefly below, and so a branch was created that was initially called
TsunamiClaw.  Later it became clear that there were many other potential
applications with similar requirements when modeling a variety of
geophysical flows over varying topography that are very shallow 
relative to their lateral extent,
and the code was generalized to become \geoclaw.  

One major issue was the desire to handle wetting and drying of grid cells at
margins of the flow, which required development of a Riemann solver that
robustly handled dry states.  The code must also be well-balanced so that an
ocean at rest remains at rest and waves with amplitudes that are very small
relative to the depth of the ocean are well resolved.  Topography appears in
the equations as a source term in the momentum equation and well balancing
requires that this source term be incorporated into the Riemann solver
rather than solving with a fractional step approach.  The Riemann solver
developed in \cite{dgeorge:phd, dgeorge:jcp} achieves these objectives.
Other features of \geoclaw include the ability to solve the equations in
latitude--longitude coordinates on the surface of the sphere, and the
incorporation of source terms modeling bottom friction using a Manning
formulation.
More details about the code and tsunami modeling applications can be found
in \cite{BergerGeorgeLeVequeMandli:awr11, LeVequeGeorgeBerger:an11}. 
In addition to a variety of tsunami modeling applications (e.g. \cite{??}),
the \geoclaw code has been used to solve dam break problems in steep terrain
\cite{George:Malpasset}, storm surge problems \cite{Mandli:ws},
and submarine landslides \cite{??}.  The code also
formed the basis for exploration for the use of 
multi-layer shallow water equations for storm surge modeling
\cite{mandli:phd}, development of storm surge applications \cite{Mandli:ws},
and is currently being extended further into D-Claw, for debris flow modeling 
\cite{Iverson:2014dc,George:2014gh}.

Roughly one third of the files in the \amrclaw source library
have to be modified for \geoclaw. 
There are 135 files at this moment 
in the \amrclaw 2D branch.
Forty-five of them are replaced by a \geoclaw-specific file of the
same name but in the \geoclaw 2D branch. 
For example, to keep a flat
sea level when interpolating, it is necessary to account for bathymetry.
So the water height plus bathymetry is interpolated rather than just
interpolating the conserved quantity, which is the default behavior in
\amrclaw.

Several substantial improvements in the
algorithms implemented in \geoclaw have been made between versions 4.6.3 and
5.3.0, including:

\begin{itemize} 
\item the ability to specify topography via a set of {\tt
topo} files that may cover overlapping regions at different resolutions. The
finite volume method requires cell averages of topography, computed by
integrating a piecewise bilinear function constructed from the input {\tt
topo} files over each grid cell.  In 5.1.0, this was improved to 
allow an arbitrary number of nested {\tt topo} grids.
When adaptive mesh refinement is used,
regridding may take place every few time steps.  Improvements were made 
in 5.2.0 so that topography could be copied rather than always being
recomputed in regions where there is an old grid at the same resolution to
copy from.  
\todo{This was included in AMRClaw above but ok to include again.}

\item The user can also provide {\tt dtopo} files that specify changes to the
initial topography at a series of times.  This is used to specify sea-floor
motion during a tsunamigenic earthquake, but can also be used to specify
submarine landslide motion or a failing dam, for example. Major changes were
made to these algorithms in 5.1.0 to fix a problem in earlier versions that
resulted in incorrect topography motion in some cases, also resulting in a much
more robust handling of multiple  {\tt dtopo} files.

\item New capabilities were added in 5.0.0 to monitor the maximum of various
flow quantities over a specified time range of a simulation.  This capability is
crucial for many applications such as tsunami hazard assessment, where the
maximum flow depth at each point in a community is desired, maximum current
velocities in a harbor, or maximum momentum flux (a measure of the hydrodynamic
force that would be exerted by the flow on a structure).  Arrival time of the
first wave at each point can also be monitored.  Some such capabilities were
included in the 4.x version of the code, but were more limited and did not
always perform properly near the edges of refinement patches.  The user can
specify a grid of points on which to monitor values, and the new code is more
flexible in allowing one-dimensional grids (e.g. a transect), two-dimensional
rectangular grids, or an arbitrary set of points.
This is described in 
\url{http://www.clawpack.org/fgmax.html}.

\item A number of new Python modules have been developed to assist the user
in working with {\tt topo} and {\tt dtopo} files.  \todo{To be continued...}

\item In depth-averaged flow the wave speed and therefore the CFL condition depends on the depth.  Thus in shallow water spatially refined grids may not need to be refined in time.  This ``variable-time-stepping'' was easily added due to the anisotropic capabilities that were added to \amrclaw.

\end{itemize} 

\todo{Recently a significant effort was made to get \geoclaw verified and validated
against the National Tsunami Hazard Mitigation Program (NTHMP) benchmarks.
After approval this allows \geoclaw to be used in hazard mapping projects that 
involve the Federal Emergency Management Agency (FEMA).
}{Is this right?}  A number of additional applications have been developed for
\geoclaw, including storm surge, debris flow, and dam breaks.
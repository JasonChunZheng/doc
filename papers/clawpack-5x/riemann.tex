%!TEX root = paper.tex
%
% Riemann
%

\subsection{Riemann: A Community-Driven Collection of Approximate Riemann
Solvers}\label{sec:riemann}

The methods implemented in \clawpack, and all modern Godunov-type methods for
hyperbolic PDEs, are based on the solution of Riemann problems as discussed
in \cref{sec:hyp}.

For nonlinear systems, the exact solution of the Riemann problem is
computationally costly and may involve both discontinuities (shocks and contact
waves) and rarefactions.  It is almost always preferable to employ inexact
Riemann solvers that approximate the solution using discontinuities only, with
an appropriate entropy condition.  The solvers available in \clawpack are
all approximate solvers, although one could easily implement their own exact
solver and make it available in the format needed by \clawpack routines.

A common feature in all packages in the \clawpack suite is the
use of a standard interface for Fortran Riemann solver routines.  This ensures that new
solvers or solver improvements developed for one package can immediately
be used by all packages.  To further facilitate this sharing and to avoid
duplication, Riemann solvers are (with rare exceptions) not maintained under
the other packages but are collected in a single repository named 
\texttt{riemann}.
Users who develop new solvers for \clawpack are encouraged to submit them to the
Riemann repository.

In the Fortran-based packages (Classic, AMRClaw, and GeoClaw) the Riemann solver
is selected at compile-time by modifying a problem-specific Makefile. In
\pyclaw, the Riemann solver to be used is selected at run-time.  This is made
possible by compiling all of the Riemann solvers (when \pyclaw is installed) and
generating Python wrappers with \texttt{f2py}.  For \pyclaw, \texttt{riemann}
also
provides metadata (such as the number of equations, the number of waves,
and the names of the conserved quantities) for each solver so that setup is
made more transparent.

Whereas most existing codes for hyperbolic PDEs use Riemann solvers to
compute fluxes, \clawpack Riemann solvers instead compute the waves
(or discontinuities) that make up the approximate Riemann solution.
\clawpack is also unusual in that it optionally makes use of {\em transverse}
Riemann solvers, responsible for computing transport between cells that
are only corner (in 2d) or edge (in 3d) adjacent.

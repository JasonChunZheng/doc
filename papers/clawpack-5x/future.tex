%!TEX root = paper.tex
%
% Future Work
%
% Lead currently:  None
%

\section{Conclusions} \label{sec:conclusions}

\clawpack has evolved over the past 20 years from its genesis as a small and
focused software package that two core developers could manage without
version control.  It is now an ecosystem of related projects that share a core
philosophy and some common code (notably Riemann solvers and visualization
tools), but that are aimed at different user
communities and that are developed by overlapping but somewhat distinct
groups of developers scattered at many institutions.  The adoption of better
software engineering practices, in particular the use of Git and GitHub as an
open development platform and the use of pull requests to discuss proposed
changes, has been instrumental in facilitating the development of many of the
new capabilities summarized in this paper.  

\subsection{Future Plans} \label{sub:future}

The \clawpack development team continues to look forward to new ideas and
efforts that will allow great accessibility to the project as well as new
capabilities that the core development team has not thought of.  To this end a
number of the broad efforts that are being considered for the next major release
of \clawpack include
\begin{itemize}
    \item An increased librarization effort with the Fortran based sub-packages,
    \item An extensible and more accessible interface to the Riemann solvers,
    \item An effort to allow \pyclaw and the \clawpack Fortran packages to rely
    on more of the same code-base,
    \item An increased emphasis on a larger development community,
    \item More support for new frameworks such as \forestclaw \cite{Burstedde:we},
    \item A refactoring of the visualization tools in \visclaw, along with
    support for additional backends, particularly for three-dimensional results
    (e.g.
\revised{
\texttt{Mayavi}\footnote{\url{http://docs.enthought.com/mayavi/mayavi/}},} 
\texttt{VisIt}\footnote{\url{https://visit.llnl.gov}}, 
\texttt{ParaView}\footnote{\url{http://www.paraview.org/}}, or 
\texttt{yt}\footnote{\url{http://yt-project.org/}}).
\end{itemize}

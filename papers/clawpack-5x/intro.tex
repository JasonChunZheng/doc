%!TEX root = paper.tex
%
% Introduction
%
% Lead currently:  Kyle Mandli
%

\section{Introduction}\label{sec:intro}

The \clawpack software suite \cite{clawpack} is designed for the solution of
nonlinear conservation laws, balance laws, and other first-order hyperbolic
partial differential equations not necessarily in conservation form.  The
underlying solvers are based on the wave propagation algorithms described by
LeVeque in \cite{rjl:fvmhp}, and are designed for logically Cartesian uniform or
mapped grids or an adaptive hierarchy of such grids.  The original \clawpack was
first released as a software package in 1994 and since then has made major
strides in both capability and interface. More recently a major refactoring of
the code and a move to GitHub for development has resulted in the release of
\clawpack 5.0 in January, 2014.  \revised{Beyond enabling a distributed and
better managed development process a number of user-facing improvements were
made including a new user interface and visualization tools, incorporation of
high-order accurate algorithms, parallelization through MPI and OpenMP, and
other enhancements.}

Because scientific software  has become central to many advances made
in science, engineering, resource management, natural hazards modeling
and other fields, it is increasingly important to describe and
document changes made to widely used packages.  Such documentation efforts
serve to orient new and existing users to the  strategies
taken by developers of the software, place the software package in the context
of other packages, document major code changes, and provide a
concrete, citable reference for users of the software.

% \newpage % to put list on page 2
With this in mind, the goals of this paper are to:

\begin{itemize}
\item Summarize the development history of \clawpack,
\item Summarize some of the major changes made between the early \clawpack
4.x versions and the most recent version, \clawpack 5.3,
\item Summarize the development model we have adopted, for
managing open source scientific software
projects with many contributors, and
\item Identify how users can contribute to the \clawpack suite of tools.
\end{itemize}

This paper provides a brief history of \clawpack in
\cref{sub:history}, a background of the mathematical concerns in \cref{sec:hyp},
the modern development approach now being used in \cref{sec:development},
the major feature additions in the \clawpack 5.x major release up until Version 5.3 in
\cref{sec:advances}. Some concluding thoughts and future plans for
\clawpack are mentioned in
\cref{sec:conclusions}.

\subsection{History of \clawpack} \label{sub:history}

The first version of \clawpack was released by LeVeque in 1994
\cite{clawpack-v1} and consisted of Fortran code for solving problems on a
single, uniform Cartesian
grid in one or two space dimensions, together with some \mlab
\cite{MATLAB:2015a} scripts
for plotting solutions. The wave-propagation method implemented
in this code provided a general way to apply recently developed
high-resolution shock capturing methods to general hyperbolic systems and
required only that the user provide a ``Riemann solver'' to specify a new
hyperbolic problem.
Collaboration with Berger \cite{mjb-rjl:amrclaw}
soon led to the incorporation of adaptive mesh refinement (AMR) in two space
dimensions, and work with Langseth \cite{jol-rjl:3d, jol:thesis}
led to three-dimensional versions of the wave-propagation algorithm and the
software, with three-dimensional AMR then added by Berger.

Version 4.3 of \clawpack contained a number of other improvements to
the code and formed the basis for the examples presented in a textbook
\cite{rjl:fvmhp} published in 2003.  That text not only provided a
complete description of the wave propagation algorithm, developed by LeVeque,
but also is notable in that the codes used to produce virtually all of figures
in the text were made available online \cite{rjl:fvmhp}.

% In 2009, \clawpack Version 4.4 was released with a major change from \mlab
% to Python as the recommended visualization tool, and the development
% of a Python user interface for specifying the input data.

In 2009, \clawpack Version 4.4 was released with a major change from \mlab to
Python as the recommended visualization tool, and the development of a Python
user interface for specifying the input data. \revised{Finally in January of
2013 the 4.x versions of \clawpack ended with the release of
4.6.3}\footnote{\revised{Details of these changes can be found at
}\url{http://depts.washington.edu/clawpack/users-4.6/changes.html}.  Version 4.x
used \texttt{svn} version control and the freely available software (under the
BSD license) was distributed via tarballs.}

Version 5 of \clawpack introduces both user-exposed features and a number of
modern approaches to code development, interfacing with other codes, and adding
new capabilities.  \revised{The move to \texttt{git} version control also
allowed a more complete open source model.}  These changes are the subject of
the rest of this paper.

\subsection{Hyperbolic problems}\label{sec:hyp}

In one space dimension, the hyperbolic systems solved with
\clawpack typically take the form of conservation laws
\begin{equation}\label{eq:hyp1a}
q_t(x,t) + f(q(x,t))_x = 0
\end{equation}
or non-conservative linear systems
\begin{equation}\label{eq:hyp1b}
q_t(x,t) + A(x) q(x,t)_x = 0,
\end{equation}
where subscripts denote partial derivatives and $q(x,t)$ is a vector with
$m\ge 1$ components.
\revised{Here the components of $q$ represent conserved quantities, while the
function $f$ represents the flux (transport) of $q$.
Equation \eqref{eq:hyp1a} generalizes in a natural way to higher
space dimensions; see the examples below.}
The coefficient matrix $A$ in \cref{eq:hyp1b} or
the Jacobian matrix $f'(q)$ in \cref{eq:hyp1a} is assumed to be
diagonalizable with real eigenvalues for all relevant values of
$q$, $x$, and $t$.  This condition guarantees that the system is hyperbolic,
with solutions that are wave-like.  The eigenvectors of the system
determine the relation between the different components
of the system, or waves, and the eigenvalues determine the speeds at which these
waves travel.  The right hand side of these equations could be
replaced by a ``source term'' $\psi(q,x,t)$ to give a non-homogeneous
equation that is sometimes called a ``balance law'' rather than a
conservation law.  Spatially-varying flux functions $f(q,x)$ in
\cref{eq:hyp1a} can also be handled using the f-wave approach
\cite{db-rjl-sm-jr:vcflux}.

Examples of equations solved by \clawpack include:
\begin{itemize}
    \item  \revised{Advection equation(s) for one or more tracers; in the simplest,
      one-dimensional case we have:
      $$q_t + (u(x,t)q)_x = 0.$$
      The velocity field $u(x,t)$ is
      typically prescribed from the solution to another fluid flow problem, such
      as wind.  Typical applications include transport of heat, energy, pollution,
      smoke, or another passively-advected quantity that does not influence the velocity field.}
    \item \revised{The shallow water equations, describing the velocity $(u,v)$ and
        surface height $h$ of a fluid whose depth is small relative to
        typical wavelengths.
        \begin{align}
            h_t + (hu)_x + (hv)_y & = 0 \\
            (hu)_t + \left(hu^2 + \frac{1}{2} g h^2\right)_x + (huv)_y & = -g b_x \\
            (hv)_t + \left(hv^2 + \frac{1}{2} g h^2\right)_y + (huv)_x & = -g b_y \\
        \end{align}
        Here $g$ is a constant related to the gravitational force and
        $b(x,y)$ is the {\em bathymetry}, or bottom surface height.
        Notice that the bathymetry enters the equations through a
        {\em source term}; additional terms could be added to model
        the effect of bottom friction.}
        These equations are used, for instance, to model inundation
        caused by tsunamis and dam breaks, as well as to model atmospheric flows.
    \item The Euler equations of compressible, inviscid fluid dynamics, consist
        of conservation laws for mass, momentum, and energy.
        The wave speeds depend on the local fluid velocity
        and the acoustic wave velocity (sound speed).  Source terms can be added
        to include the effect of gravity, viscosity or heat transfer.
        These systems have important applications in
        aerodynamics, climate and weather modeling, and astrophysics.
    \item Elastic wave equations, used to model 
        \revised{compressional and shear} waves in solid
        materials.  \revised{Here even linear models can be complex due to varying
        material properties on multiple scales that affect the wave speeds
        and eigenvectors.}
\end{itemize}
%For a one-dimensional problem \cref{eq:hyp1a} or \cref{eq:hyp1b},
%the {\em Riemann problem} consists of the equation together with
%piecewise constant initial data with a single jump discontinuity.

Discontinuities (shock waves) can arise in the solution of nonlinear
hyperbolic equations, causing difficulties for traditional numerical
methods based on discretizing derivatives directly.  Modern shock
capturing methods are often based on solutions to the {\em Riemann
problem} that consists of equations \cref{eq:hyp1a} or
\cref{eq:hyp1b} together with piecewise constant initial data with a
single jump discontinuity.  The solution to the Riemann problem is a
similarity solution (a function of $x/t$ only), typically consisting
of $m$ waves (for a system of $m$ equations) propagating at constant
speed.  This is true even for nonlinear problems, where the waves may
be shocks or rarefaction waves (through which the
solution varies continuously in a self-similar manner).

The main theoretical and numerical difficulties of hyperbolic problems
involve the prescription of physically correct weak solutions and
understanding the behavior of the solution at discontinuities.  The
Riemann solver is an algorithm that encodes the specifics of the
hyperbolic system to be solved, and it is the only routine (other than
problem-specific setup such as initial conditions) 
that needs to be changed in order to apply the
code to different hyperbolic systems.  In some cases, the Riemann
solver may also be designed to enforce physical properties like
positivity (e.g., for the water depth in \geoclaw) or to account for
forces (like that of gravity) that may be balanced by flux terms.

\clawpack is based on Godunov-type finite volume methods in which
the solution is represented by cell averages.  Riemann problems
between the cell averages in neighboring states are used as the
fundamental building block of the algorithm.
The wave-propagation algorithm originally
implemented in \clawpack (and still used in much of the code) is based on
using the waves resulting from each Riemann solution together with limiter
functions to achieve second-order accuracy where the solution is smooth
together with sharp resolution of discontinuities without spurious numerical
oscillations (see \cite{rjl:fvmhp} for a detailed description of the
algorithms).   \revised{Higher-order WENO methods have also been developed
relying on the same Riemann solvers.  These methods can be found in \pyclaw 
(see \cref{sec:pyclaw}), one of the packages in the larger \clawpack ecosystem.}

Problem-specific boundary conditions must also be imposed, which
are implemented by a subroutine that sets the solution value in
{\em ghost cells} exterior to the domain each time step.  The
\clawpack software contains library routines that implement several
sets of boundary conditions that are commonly used, {\em e.g.}
periodic boundary conditions, reflecting solid wall boundary
conditions for problems such as acoustics, Euler, or shallow water
equations, and non-reflecting (absorbing) extrapolation boundary
conditions.  As with all \clawpack library routines, the boundary
condition routine can be copied and modified by the user to implement
other boundary conditions needed for a particular application.

In two or three space dimensions, the wave-propagation methods
are extended using either dimensional splitting, so that only
one-dimensional Riemann solvers are needed, or by a multi-dimensional
algorithm based on {\em transverse Riemann solvers} introduced in
\cite{rjl:wpalg}.  Both approaches are supported in \clawpack.
A variety of Riemann solvers have been developed for \clawpack, many
of which are collected in the \texttt{riemann} repository, see
\cref{sec:riemann}.

Adaptive mesh refinement (AMR) is essential for many problems and has been
available in two space dimensions since 1995, when Marsha Berger
joined the project team and her AMR code for the Euler equations of
compressible flow was generalized to fit into the software which
became \amrclaw \cite{Berger:1998ia}, \revised{another package included in the
\clawpack ecosystem.} \amrclaw was carried over to
three space dimensions using the unsplit algorithms introduced in
\cite{jol-rjl:3d}.  Starting in Version 5.3.0, dimensional splitting
is also supported in \amrclaw, which can be particularly useful in
three space dimensions where the unsplit algorithms are much more
expensive.  Other recent improvements to \amrclaw are discussed in
\cref{sec:amrclaw}.

\revised{
There are several other open source software projects that provide adaptive
mesh refinement for hyperbolic PDEs.  The interested reader may want to
investigate AMROC \cite{Deiterding:2011bj}, 
BoxLib\footnote{\url{https://ccse.lbl.gov/BoxLib/index.html}}, 
Chombo \cite{chombo}, Gerris \cite{gerris}, OpenFOAM \cite{OpenFOAM}, 
or SAMRAI \cite{samrai}, for example.}
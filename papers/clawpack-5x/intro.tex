%!TEX root = paper.tex
%
% Introduction
%
% Lead currently:  Kyle Mandli
%

\section{Introduction}\label{sec:intro}

The \clawpack software suite \cite{clawpack} is designed for the solution of
nonlinear conservation laws, balance laws, and other first-order hyperbolic
partial differential equations not necessarily in conservation form.  \clawpack
was first released in 1994 and since then has made major strides in both
capability and interface. More recently a major refactoring of the code and a
move to GitHub for development has resulted in the release of
\clawpack 5.0 in January, 2014. A significant number of additional improvements
have been made since then. The goals of this paper are to:
\begin{itemize}
\item summarize some of the major changes made between the \clawpack
4.x codes and the most recent \clawpack 5.3,
\item summarize the development model we have adopted, which we believe
may be of interest to others managing open source scientific software
projects with many contributors, and
\item encourage users to contribute to the \clawpack suite of tools.
\item provide a citable reference for \clawpack 5.x that includes
substantial contributors to this code as authors,
\end{itemize}
Also, by submitting this to a refereed journal we are trying to elevate
the role of producing software in academic departments, and we
would like to acknowledge the substantial contributions to the
code by the many authors.

To meet these aims, this paper provides a brief history of \clawpack in 
\cref{sub:history}, a background of the mathematical concerns in \cref{sec:hyp},
the modernized development approach now being used in \cref{sec:development},
the major feature additions in the \clawpack 5.x major release up until 5.3 in
\cref{sec:advances}, and finally some concluding thoughts and future plans for
\clawpack in
\cref{sec:conclusions}.

\subsection{History of \clawpack} \label{sub:history}

The first version of \clawpack was released by LeVeque in 1994
\cite{clawpack-v1} and consisted of Fortran code for solving problems on a
fixed grid in one or two space dimensions, together with some Matlab scripts
for plotting solutions. The wave-propagation method implemented
in this code provided a general way to apply recently developed
high-resolution shock capturing methods to general hyperbolic systems and
required only that the user provide a ``Riemann solver'' to specify a new
hyperbolic problem.
Collaboration with Berger \cite{mjb-rjl:amrclaw}
soon led to the incorporation of adaptive mesh refinement (AMR) in two space
dimensions, and work with Langseth \cite{jol-rjl:3d, jol:thesis}
led to three-dimensional versions of the wave-propagation algorithm and the
software, with three-dimensional AMR then added by Berger.

Version 4.3 of \clawpack contained a number of other improvements to
the code and formed the basis for the examples presented in a textbook
\cite{rjl:fvmhp} published in 2003.  That text not only provided a
complete description of the wave propagation algorithm, developed by LeVeque,
but also is notable in that the codes used to produce virtually all of figures
in the text were made available online \cite{rjl:fvmhp}
at \url{http://depts.washington.edu/clawpack/clawpack-4.3/book.html}.

In 2009, \clawpack Version 4.4 was released with a major change from Matlab
to Python as the recommended visualization tool, and the development
of a Python user interface for specifying the input data.

Since then, a number of other features were added to better cope with new
applications, to provide a better user interface and visualization tools, and to
incorporate higher-order accurate algorithms, MPI and OpenMP parallelization, and
other enhancements. The \clawpack 4.x line of code ended with Version 4.6.3
(released in January 2013) with many of the changes from 4.3 to 4.6 summarized at
\url{http://depts.washington.edu/clawpack/users-4.6/changes.html}.

Version 5 of \clawpack introudces a number of modern
approaches to code development, interfacing with other codes,  and
adding new capabilities. This is 
the subject of the rest of the paper.

\subsection{Hyperbolic problems}\label{sec:hyp}

In one space dimension, the hyperbolic systems solved with
\clawpack typically take the form of conservation laws
\begin{equation}\label{eq:hyp1a}
q_t(x,t) + f(q(x,t))_x = 0
\end{equation}
or non-conservative linear systems
\begin{equation}\label{eq:hyp1b}
q_t(x,t) + A(x) q(x,t)_x = 0,
\end{equation}
where subscripts denote partial derivatives and $q$ is a vector with
$m\ge 1$ components.  The coefficient matrix $A$ in \cref{eq:hyp1b} or
the Jacobian matrix $f'(q)$ in \cref{eq:hyp1a} is assumed to be
diagonalizable with real eigenvalues for all relevant values of
$q,~x,~t$.  This condition guarantees that the system is hyperbolic,
with solutions that are wave-like.  The eigenvectors of the system
determine the relation between the different components
of system, or waves, and the eigenvalues determine the speeds at which these
waves travel.  The right hand side of these equations could be
replaced by a ``source term'' $\psi(q,x,t)$ to give a non-homogeneous
equation that is sometimes called a ``balance law'' rather than a
conservation law.  Spatially-varying flux functions $f(q,x)$ in
\cref{eq:hyp1a} can also be handled using the f-wave approach
\cite{db-rjl-sm-jr:vcflux}.

Some common examples of such hyperbolic systems include:
\begin{itemize}
    \item The Euler equations of compressible, inviscid fluid dynamics.
        They consist of conservation laws for mass, momentum, and energy.
        The wave speeds are a combination of the local fluid velocity
        and the acoustic wave velocity.  Source terms have been added
        to include the effect of gravity, or could be added to include 
        viscosity and heat transfer.
        These systems have important applications in
        aerodynamics, climate and weather modeling, and astrophysics.
    \item The shallow water equations, describing the velocity and
        surface height of a liquid whose depth is small relative to
        typical wavelengths.  In this case source terms may include
        the effect of varying bathymetry and of bottom friction.
        These equations are used, for instance, to model inundation
        caused tsunamis and dam breaks, as well as atmospheric flows.
    \item Elastic wave equations, used to model waves in solid
    materials. 
\end{itemize}
These examples are part of Clawpack's sample applications.
%For a one-dimensional problem \cref{eq:hyp1a} or \cref{eq:hyp1b},
%the {\em Riemann problem} consists of the equation together with
%piecewise constant initial data with a single jump discontinuity.

Discontinuities (shock waves) can arise spontaneously in the solution of nonlinear
hyperbolic equations, causing difficulties for
traditional numerical methods based on discretizing derivatives
directly.   Modern shock capturing methods are generally based on solutions
to the {\em Riemann problem}, which consists of the equation
\cref{eq:hyp1a} or \cref{eq:hyp1b}
together with piecewise constant initial data with a single jump discontinuity.
The solution to the Riemann problem is a similarity
solution (a function of $x/t$ only),
typically consisting of $m$ waves (for a system of $m$ equations)
propagating at constant speed.  This is true even for nonlinear problems,
where the waves may be discontinuous shocks or rarefaction waves
(through which the solution varies continuously in a self-similar manner).

The main theoretical
and numerical difficulties of hyperbolic problems involve the prescription of
physically correct weak solutions -- in other words, understanding the behavior
of the solution at discontinuities.  The Riemann solver is an algorithm that
encodes the specifics of the hyperbolic system to be solved, and it is the only
routine (other than problem-specific setup) that needs to be changed in order
to apply the code to different hyperbolic systems.  In some cases, the Riemann
solver may also be designed to enforce physical properties like positivity
(e.g., for the water depth in \geoclaw) or to account for forces (like that
of gravity) that may be balanced by flux terms.
In practice, approximate Riemann solvers are commonly used for nonlinear
problems.%, which replace rarefaction waves by one or more discontinuities.

\clawpack is based on Godunov-type finite volume methods in which
the solution is represented by cell averages.  Riemann problems
between the cell averages in neighboring states are used as the
fundamental building block of the algorithm.
The wave-propagation algorithm originally
implemented in \clawpack (and still used in much of the code) is based on
using the waves resulting from each Riemann solution together with limiter
functions to achieve second-order accuracy where the solution is smooth
together with sharp resolution of discontinuities without spurious numerical
oscillations (see \cite{rjl:fvmhp} for a detailed description of the
algorithms).   The more-recently developed SharpClaw algorithms (see
\cref{sec:pyclaw}), now incorporated in \pyclaw, use higher-order WENO methods
but rely on the same Riemann solvers.

In two space dimensions, hyperbolic equations might take the form
\begin{equation}\label{eq:hyp2a}
q_t(x,y,t) + f(q(x,y,t))_x + g(q(x,y,t))_y = 0
\end{equation}
or
\begin{equation}\label{eq:hyp2b}
q_t(x,y,t) + A(x,y)q(x,y,t)_x + B(x,y)q(x,y,t)_y = 0
\end{equation}
In order to be hyperbolic, the coefficient matrices $A$ and $B$ in \cref{eq:hyp2a}
or the Jacobian matrices $f'(q)$ and $g'(q)$ in \cref{eq:hyp2b} 
must have the property that any linear
combination gives a diagonalizable matrix with real eigenvalues.  The
extension to three space dimensions is similar. 

In two or three space dimensions, the wave-propagation methods
are extended using either dimensional splitting, so that only
one-dimensional Riemann solvers are needed, or by a multi-dimensional
algorithm based on {\em transverse Riemann solvers} introduced in
\cite{rjl:wpalg}.  Both approaches are supported in \clawpack.
A variety of Riemann solvers have been developed for \clawpack, many
of which are collected in the \texttt{riemann} repository, see
\cref{sec:riemann}.  

Adaptive mesh refinement (AMR) is essential for many problems and
has been available in two space dimensions
since 1995, when Marsha Berger joined the project team and her AMR
code for the Euler equations of compressible flow was generalized to fit into
the software as \amrclaw, as described in \cite{Berger:1998ia}.  This was
later carried over to three space dimensions using the unsplit algorithms
introduced in \cite{jol-rjl:3d}.  Starting in Version 5.3.0, dimensional
splitting is also supported in \amrclaw, which can be particularly useful in
three space dimensions where the unsplit algorithms are much more expensive
due to the transverse Riemann solves.  Other recent improvements to \amrclaw
are discussed in \cref{sec:amrclaw}.

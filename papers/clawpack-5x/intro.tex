%!TEX root = paper.tex
%
% Introduction
%
% Lead currently:  Kyle Mandli
%

\section{Introduction}\label{sec:intro}

The \clawpack software suite \cite{clawpack}
is designed for the solution of nonlinear conservation laws, balance laws,
and other first-order hyperbolic partial differential
equations that need not be in conservation form. 
A brief overview of such problems is given in \cref{sec:hyp}.

The first version of \clawpack was released by LeVeque in 1994 
\cite{clawpack-v1} and consisted of Fortran code for solving problems on a
fixed grid in one or two space dimensions, together with some Matlab scripts
for plotting solutions. The wave-propagation method implemented
in this code provided a general way to apply recently developed
high-resolution shock capturing methods to general hyperbolic systems,
requiring only that the user provide a ``Riemann solver'' to specify a new
hyperbolic problem, as discussed further in \cref{sec:riemann}.  
Collaboration with Berger \cite{mjb-rjl:amrclaw} 
soon led to the incorporation of adaptive mesh refinement (AMR) in two space
dimensions, and work with Langseth \cite{jol-rjl:3d, jol:thesis}
led to three-dimensional versions of the wave-propagation algorithm and the
software, with three-dimensional AMR then added by Berger.

Version 4.3 of \clawpack contained a number of other improvements to the code
and formed the basis for the examples presented in a textbook \cite{rjl:fvmhp}
published in 2003, for which virtually all of the figures were
produced using codes available at
\url{http://depts.washington.edu/clawpack/clawpack-4.3/book.html}.

In 2009, Version 4.4 was released with a major change from Matlab
to Python as the recommended visualization tool, and the development
of a Python user interface for specifying the input data.

Since then, a number of other features were added to better cope with new
applications, to provide a better user interface and visualization tools, and to
incorporate higher-order accurate algorithms, OpenMP parallelization, and
other enhancements. The \clawpack 4.x line of code ended with Version 4.6.3
(released in January, 2013) and
many of the changes from 4.3 to 4.6 are summarized at
\url{http://depts.washington.edu/clawpack/users-4.6/changes.html}.

Major refactoring of the code and a move to GitHub for development over the
next year resulted in the release of \clawpack 5.0 in January, 2014. 
A significant number of additional improvements have been made since then.  The
objectives of this paper are to:
\begin{itemize} 
\item summarize some of the major changes made between the \clawpack
4.x codes and the most recent \clawpack 5.3,
\item provide a citable reference for \clawpack 5.x that includes
substantial contributors to this code as authors, and
\item summarize the development model we have adopted, which we believe
may be of interest to others managing large open source scientific software
projects.
\end{itemize} 

\subsection{Hyperbolic problems}\label{sec:hyp}

In one space dimension, the hyperbolic equation solved with
\clawpack typically takes the form of a conservation law
\begin{equation}\label{eq:hyp1a}
q_t(x,t) + f(q(x,t))_x = 0
\end{equation}
or a non-conservative linear equation
\begin{equation}\label{eq:hyp1b}
q_t(x,t) + A(x) q(x,t)_x = 0,
\end{equation} 
where subscripts denote partial derivatives and
$q$ could be a vector with $m$ components for a system of equations. 
The coefficient matrix $A$ in \cref{eq:hyp1b} or the Jacobian matrix 
$f'(q)$ in \cref{eq:hyp1a} is assumed to be diagonalizable with real eigenvalues
for all relevant values of $q,~x,~t$. 
This condition
guarantees that the system is hyperbolic, with solutions that are wave-like.  
The eigenvalues are the wave speeds.  For a system of equations,
the eigenvectors determine the relation between different components of the
system in waves of different families.  The right hand side of these
equations could be replaced by a ``source term'' $\psi(q,x,t)$ to give a
non-homogeneous equation that is sometimes called a ``balance law'' rather
than a conservation law in the case of \cref{eq:hyp1a}.
Spatially-varying flux functions $f(q,x)$ in \cref{eq:hyp1a} can also be handled
using the f-wave approach \cite{db-rjl-sm-jr:vcflux}.

%For a one-dimensional problem \cref{eq:hyp1a} or \cref{eq:hyp1b},
%the {\em Riemann problem} consists of the equation together with
%piecewise constant initial data with a single jump discontinuity.

Discontinuities can arise spontaneously in the solution of nonlinear
hyperbolic equations (shock wave formation), causing difficulties for
traditional finite-difference methods based on discretizing derivatives
directly.   Modern shock capturing methods are generally based on solutions
to the {\em Riemann problem}, which consists of the equation 
\cref{eq:hyp1a} or \cref{eq:hyp1b}
together with piecewise constant initial data with a single jump discontinuity.
The solution to the Riemann problem is a similarity
solution (a function of $x/t$ only),
typically consisting of $m$ waves (for a system of $m$ equations)
propagating at constant speed.  This is true even for nonlinear problems,
where the waves may be discontinuous shocks or rarefaction waves
(through which the solution varies continuously in a self-similar manner).

The main theoretical
and numerical difficulties of hyperbolic problems involve the prescription of
physically correct weak solutions -- in other words, understanding the behavior
of the solution at discontinuities.  The Riemann solver is an algorithm that
encodes the specifics of the hyperbolic system to be solved, and it is the only
routine (other than problem-specific setup) that needs to be changed in order
to apply the code to different hyperbolic systems.  In some cases, the Riemann
solver may also be designed to enforce physical properties like positivity
(e.g., for the water depth in \geoclaw) or to account for forces (like that
of gravity) that may be balanced by flux terms.
In practice approximate Riemann solvers are commonly used for nonlinear
problems.%, which replace rarefaction waves by one or more discontinuities.

\clawpack is based on Godunov-type finite volume methods in which
the solution is represented by cell averages.  Riemann problems
between the cell averages in neighboring states are used as the
fundamental building block of the algorithm.
The wave-propagation algorithms originally
implemented in \clawpack (and still used in much of the code) is based on
using the waves resulting from each Riemann solution together with limiter
functions to achieve second-order accuracy where the solution is smooth
together with sharp resolution of discontinuities without spurious numerical
oscillations (see \cite{rjl:fvmhp} for a detailed description of the
algorithms).   The more-recently developed SharpClaw algorithms,
now incorporated in
\pyclaw, use higher-order WENO methods, but still rely on the same Riemann
solvers (see \cref{sec:pyclaw}).  

In two space dimensions, hyperbolic equations might take the form
\begin{equation}\label{eq:hyp2a}
q_t(x,y,t) + f(q(x,y,t))_x + g(q(x,y,t))_y = 0
\end{equation}
or
\begin{equation}\label{eq:hyp2b}
q_t(x,y,t) + A(x,y)q(x,y,t)_x + B(x,y)q(x,y,t)_y = 0
\end{equation} 
In this case the
coefficient matrices $A$ and $B$ in \cref{eq:hyp2a} or the Jacobian matrices
$f'(q)$ and $g'(q)$ in \cref{eq:hyp2b} are assumed to be hyperbolic, i.e. have
the property that any
linear combination gives a diagonalizable matrix with real eigenvalues.
The extension to three space dimensions is similar.

In two or three space dimensions, the wave-propagation methods
are extended using either dimensional splitting, so that only
one-dimensional Riemann solvers are needed, or by a multi-dimensional
algorithm based on {\em transverse Riemann solvers} introduced in 
\cite{rjl:wpalg}.  Both approaches are supported in \clawpack.

A variety of Riemann solvers have been developed for \clawpack, many of which
are collected in the \texttt{riemann} repository, see \cref{sec:riemann}.

